% Created 2025-07-24 jue 22:30
% Intended LaTeX compiler: pdflatex
\documentclass[presentation,8pt, aspectratio=169, compress, xcolor=svgnames]{beamer}
\usepackage[utf8]{inputenc}
\usepackage[T1]{fontenc}
\usepackage{graphicx}
\usepackage{longtable}
\usepackage{wrapfig}
\usepackage{rotating}
\usepackage[normalem]{ulem}
\usepackage{amsmath}
\usepackage{amssymb}
\usepackage{capt-of}
\usepackage{hyperref}
\setbeamertemplate{navigation symbols}{}
\usepackage[english]{babel}
\usepackage[utf8]{inputenc}
\usepackage{minted}
\usepackage{tikz}
\usemintedstyle{emacs}
\usepackage{color}
\usepackage{animate}
\usepackage{subcaption}
\newcommand{\shorttitle}{Bogotá - Colombia}
\newcommand{\shortauthor}{Vinasco-Salinas}
\institute{Instituto Geográfico Agustín Codazzi, Dirección de Investigación y Prospectiva,  Bogotá, COLOMBIA}
%\usepackage[type={CC}, modifier={by-sa}, version={4.0},]{doclicense}
\titlegraphic{\includegraphics[width=0.2\textwidth]{Artwork/logo-igac-colorhorizontal.png}}
\usepackage[usenames,dvipsnames]{xcolor}
\usepackage{tikz} \usetikzlibrary{calc, arrows.meta, intersections, patterns, positioning, shapes.misc, fadings, through,decorations.pathreplacing}
\definecolor{ColorOne}{named}{MidnightBlue}
\definecolor{ColorTwo}{named}{ForestGreen}
\definecolor{ColorThree}{named}{Turquoise}
\tikzstyle{descript} = [text = black,align=center, minimum height=1.8cm, align=center, outer sep=0pt,font = \footnotesize]
\tikzstyle{activity} =[align=center,outer sep=1pt]
\usetheme{Cesbio}
\author{Juan Sebastian Vinasco Salinas}
\date{\today}
\title{Principios y aplicaciones de las imágenes hiperespectrales}
\subtitle{Una introducción}
\hypersetup{
 pdfauthor={Juan Sebastian Vinasco Salinas},
 pdftitle={Principios y aplicaciones de las imágenes hiperespectrales},
 pdfkeywords={},
 pdfsubject={},
 pdfcreator={Emacs 30.1 (Org mode 9.7.19)}, 
 pdflang={English}}
\usepackage[backend=bibtex,style=ieee]{biblatex}
\addbibresource{/home/juanse/Documents/Proyectos/IGAC_Diplomado/referencias/referecias_diplomado.bib}
\begin{document}

\maketitle
\begin{frame}{Outline}
\tableofcontents
\end{frame}

\section{Introducción}
\label{sec:org4a6d255}
\begin{frame}[label={sec:orgf6e0749}]{Recorderis del Espectro Electromagnético}
Sera \autocite{brosinsky2022beyond}
\end{frame}
\begin{frame}[label={sec:org92aa233}]{Resolución Espectral}
\end{frame}
\section{Imágenes Hiperespectrales}
\label{sec:org8529fbe}

\begin{frame}[label={sec:orgccc4824}]{Pero, ¿Que es la teledetección hiperespectral?}
\end{frame}
\begin{frame}[label={sec:orgbbea056}]{Multi-Espectra versus Hiperespectral}
\end{frame}
\section{Aplicaciones}
\label{sec:org13d86f6}
\begin{frame}[label={sec:org336c990}]{Aplicaciones de las imágenes hiperespectrales}
\begin{columns}
\begin{column}{0.5\columnwidth}
\begin{quotation} %% 
\begin{figure}[htbp]
\centering
\includegraphics[width=.9\linewidth]{./Artwork/learning_material_overview_757.jpeg}
\caption{\label{fig:rx_score}Algunas aplicaciones de las imágenes hiperespectrales}
\end{figure}
\end{quotation}
\end{column}
\begin{column}{0.5\columnwidth}
\begin{quotation} %% 
\begin{itemize}
\item Agricultura
\item Geología
\item Bosques
\item Nieve
\item Urbano
\end{itemize}
\end{quotation}
\end{column}
\end{columns}
\end{frame}
\section{Fuentes de datos}
\label{sec:orgf4f838e}
\begin{frame}[label={sec:org595d910}]{Fuentes de datos hiperespectrales}
\end{frame}
\section{Técnicas de procesamiento}
\label{sec:org03f6199}
\begin{frame}[label={sec:org4bbe15d}]{Procesamiento de datos hiperespectrales}
\end{frame}
\begin{frame}[label={sec:orgda93d54}]{Maldición de la dimensionalidad}
\end{frame}
\begin{frame}[label={sec:orgf949a03}]{Técnicas clasicas de procesamiento de imágenes espectrales}
\end{frame}
\begin{frame}[label={sec:orgb14a5dc}]{Manos a la obra}
\end{frame}
\begin{frame}[label={sec:org5375d50}]{Técnicas de aprendizaje en las imagenes espectrales}
\end{frame}
\begin{frame}[label={sec:org6e0d7c3}]{Manos a la obra}
\end{frame}
\begin{frame}[label={sec:orgdf277d9}]{Referencias}
\printbibliography[heading=none]

\newpage
\end{frame}
\end{document}
