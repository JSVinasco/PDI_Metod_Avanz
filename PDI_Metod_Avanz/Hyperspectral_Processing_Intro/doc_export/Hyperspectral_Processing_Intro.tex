% Created 2025-07-24 jue 22:27
% Intended LaTeX compiler: pdflatex
\documentclass[12pt]{article}
\usepackage[utf8]{inputenc}
\usepackage[T1]{fontenc}
\usepackage{graphicx}
\usepackage{longtable}
\usepackage{wrapfig}
\usepackage{rotating}
\usepackage[normalem]{ulem}
\usepackage{amsmath}
\usepackage{amssymb}
\usepackage{capt-of}
\usepackage{hyperref}
\usepackage[spanish]{babel}
\usepackage{lmodern} % Ensures we have the right font
\usepackage[utf8]{inputenc}
\usepackage{graphicx}
\usepackage{amsmath, amsthm, amssymb}
\usepackage[table, xcdraw]{xcolor}
\usepackage{listings}
\usepackage{times}
\usepackage[margin=0.79in]{geometry}
\usepackage{setspace}
\spacing{1.5}
\renewcommand\maketitle{\begin{titlepage}
\thispagestyle{empty}
\begin{center}
\vspace*{1cm}
\hrule
\\[0.5cm]\Large\textsc{ Introducción a el procesamiento de Imágenes Hiperespectrales}\\[0.5cm]
\\[0.5cm]\large\textsc{ Diplomado en métodos avanzados en procesamiento digital de imágenes }\\[0.5cm]
\includegraphics[width=\textwidth]{./Artwork/calamar_julio_2025.png}
\\*[0.5cm]
\hrule
\\*[0.5cm]
\large\today
\\*[1cm]
Juan Sebastian Vinasco Salinas\\
\\*[1cm]
Dirección de Investigación y Prospectiva\\
\vfilL
\includegraphics[height=3cm]{./Artwork/logo-igac-colorhorizontal.png}\\*[1cm]
\end{center}
\end{titlepage}}
\author{Juan Sebastian Vinasco Salinas}
\date{\today}
\title{Introduction to Hyperspectral Processing}
\hypersetup{
 pdfauthor={Juan Sebastian Vinasco Salinas},
 pdftitle={Introduction to Hyperspectral Processing},
 pdfkeywords={},
 pdfsubject={},
 pdfcreator={Emacs 30.1 (Org mode 9.7.19)}, 
 pdflang={Spanish}}
\usepackage[backend=bibtex,style=ieee]{biblatex}
\addbibresource{/home/juanse/Documents/Proyectos/IGAC_Diplomado/referencias/referecias_diplomado.bib}
\begin{document}

\maketitle
\setcounter{tocdepth}{2}
\tableofcontents

\lstlistoflistings
\listoftables

\renewcommand{\listfigurename}{Lista de figuras}
\listoffigures


\newpage


\newpage
\section{Teoría}
\label{sec:org04d0c4b}

Conjuntos de datos hiperespectrales, como lo son los datos abiertos de Enmap o\footnote{\href{https://www.enmap.org/data\_tools/exampledata/}{Example Data Products - EnMAP}}  el conjunto de datos de VANT de la ciudad de Toulouse \footnote{\href{https://www.toulouse-hyperspectral-data-set.com/\#download}{Toulouse Hyperspectral Data Set}}

Este material esta fuertemente inspirado en los materiales de ARSET \footnote{\url{https://appliedsciences.nasa.gov/sites/default/files/2021-02/Hyperspectral\_Part1\_Final.pdf}} y la capacitación del gobierno alemán sobre los datos de la misión Enmap \footnote{\href{https://eo-college.org/courses/beyond-the-visible-imaging-spectroscopy-for-agricultural-applications/}{Beyond the Visible – Imaging Spectroscopy for Agricultural Applications – EO \ldots{}}}
\subsection{Recorderis del Espectro Electromagnético}
\label{sec:org70b1ceb}

\subsubsection{Definición}
\label{sec:org776000f}
\subsubsection{Carterísitcas del espectro electromagnético}
\label{sec:org5e7d236}
\subsection{Resolución Espectral}
\label{sec:org538d454}

\subsubsection{Defición}
\label{sec:org86c5840}
\subsubsection{Ejemplos}
\label{sec:orgb7302d2}


\begin{figure}[htbp]
\centering
\includegraphics[width=.9\linewidth]{./Artwork/reponses_Venµs.png}
\caption{\label{figmulti_vs_hiper}Respuesta Espectral de las bandas del satelite  Venµs}
\end{figure}
\subsection{Pero, ¿Que es la teledetección hiperespectral?}
\label{sec:orga72cae7}


\subsection{Multi-Espectra versus Hiperespectral}
\label{sec:org891cc9c}

\begin{figure}[htbp]
\centering
\includegraphics[width=.9\linewidth]{./Artwork/multiespectral_vs_hiperespectral.png}
\caption{\label{figmulti_vs_hiper}Multi-espectral versus Hiperespectral}
\end{figure}
\subsection{Aplicaciones de las imágenes hiperespectrales}
\label{sec:org095395f}


\subsection{Fuentes de datos hiperespectrales}
\label{sec:org51d2d16}
\textbf{**}
\subsection{Procesamiento de datos hiperespectrales}
\label{sec:org8a9fea0}


\subsection{Maldición de la dimensionalidad}
\label{sec:org5987d28}


\subsection{Técnicas clasicas de procesamiento de imágenes espectrales}
\label{sec:org6590a76}


\subsection{Ejercicio de desmezclado de pixeles}
\label{sec:org748d2fb}

*Este ejercicio ha sido tomado de la documentación de orfeo toolbox


\begin{figure}[htbp]
\centering
\includegraphics[width=.9\linewidth]{./Artwork/cuprite_rgb.png}
\caption{\label{fig:cuprita}Teniendo en cuenta la imagen de cuprita, aquí se muestran las bandas 16, 100 y 180.}
\end{figure}


\begin{verbatim}
otbcli_EndmemberNumberEstimation  -in inputImage.tif
                                  -algo vd
                                  -algo.vd.far 1e-5

otbcli_VertexComponentAnalysis  -in inputImage.tif
                                -ne 19
                                -outendm endmembers.tif
\end{verbatim}



\begin{figure}[htbp]
\centering
\includegraphics[width=.9\linewidth]{./Artwork/hyperspectralUnmixing_rgb.png}
\caption{\label{fig:desmeclado}Imagen sin mezclar resultante, aquí se muestran las tres primeras bandas.}
\end{figure}
\subsection{Ejercicio clasificación basado en datos espectrales}
\label{sec:org2304e68}


\begin{verbatim}
otbcli_SpectralAngleClassification -in inputImage.tif
                                   -ie endmembers.tif
                                   -out classification.tif
                                   -mode sam
\end{verbatim}



\begin{figure}[htbp]
\centering
\includegraphics[width=.9\linewidth]{./Artwork/classification.png}
\caption{\label{fig:sam_clasif}Resultado de una clasificación tipo SAM (Spectral Angle Mapper)}
\end{figure}
\subsection{Ejercicio de detección de anomalías}
\label{sec:orgc135c7c}

\begin{figure}[htbp]
\centering
\includegraphics[width=.9\linewidth]{./Artwork/pavia.png}
\caption{\label{fig:pavia}Imagen hiperespectral de la Universidad de Pavía, aquí se muestran las bandas 50, 30 y 10.}
\end{figure}





\begin{verbatim}
otbcli_DimensionalityReduction  -in inputImage.tif
                                -out reducedData.tif
                                -method pca

otbcli_LocalRxDetection  -in reducedData.tif
                         -out RxScore.tif
                         -ir 1
                         -er

otbcli_BandMath  -il RxScore.tif
                 -out anomalyMap.tif
                 -exp "im1b1>100"
\end{verbatim}


\begin{figure}[htbp]
\centering
\includegraphics[width=.9\linewidth]{./Artwork/rx_score.png}
\caption{\label{fig:rx_score}Puntuación Rx calculada}
\end{figure}



\begin{figure}[htbp]
\centering
\includegraphics[width=.9\linewidth]{./Artwork/rx_detection.png}
\caption{\label{fig:rx_detection}Anomalías detectadas (en rojo)}
\end{figure}
\subsection{Técnicas de aprendizaje en las imagenes espectrales}
\label{sec:orgaee8954}


\subsection{Manos a la obra}
\label{sec:org3f6606f}
\section{Trabajo autonomo}
\label{sec:orgc1311f2}

Realice los siguientes ejercicios:

\begin{enumerate}
\item Realice el ejercicio de desmezclado de pixeles sobre las imágenes de muestra de los datos Enmap en la localización de su preferencia
\item Realice el ejercicio de clasificación basado en datos espectrales sobre las imágenes de muestra de los datos Enmap en la localización de su preferencia
\item Realice el ejercicio de detección de anomalías sobre las imágenes de muestra de los datos Enmap en la localización de su preferencia
\end{enumerate}
\section*{Referencias}
\label{sec:orge7b5a50}
\printbibliography[heading=none]

\newpage
\end{document}
