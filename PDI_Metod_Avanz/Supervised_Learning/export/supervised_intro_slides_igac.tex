% Created 2025-08-11 lun 14:19
% Intended LaTeX compiler: pdflatex
\documentclass[presentation,8pt, aspectratio=169, compress, xcolor=svgnames]{beamer}
\usepackage[utf8]{inputenc}
\usepackage[T1]{fontenc}
\usepackage{graphicx}
\usepackage{longtable}
\usepackage{wrapfig}
\usepackage{rotating}
\usepackage[normalem]{ulem}
\usepackage{amsmath}
\usepackage{amssymb}
\usepackage{capt-of}
\usepackage{hyperref}
\setbeamertemplate{navigation symbols}{}
\usepackage[english]{babel}
\usepackage[utf8]{inputenc}
\usepackage{minted}
\usepackage{tikz}
\usemintedstyle{emacs}
\usepackage{color}
\usepackage{animate}
\usepackage{subcaption}
\newcommand{\shorttitle}{Bogotá - Colombia}
\newcommand{\shortauthor}{Vinasco-Salinas}
\institute{Instituto Geográfico Agustín Codazzi, Dirección de Investigación y Prospectiva,  Bogotá, COLOMBIA}
%\usepackage[type={CC}, modifier={by-sa}, version={4.0},]{doclicense}
\titlegraphic{\includegraphics[width=0.2\textwidth]{Artwork/logo-igac-colorhorizontal.png}}
\usepackage[usenames,dvipsnames]{xcolor}
\usepackage{tikz} \usetikzlibrary{calc, arrows.meta, intersections, patterns, positioning, shapes.misc, fadings, through,decorations.pathreplacing}
\definecolor{ColorOne}{named}{MidnightBlue}
\definecolor{ColorTwo}{named}{ForestGreen}
\definecolor{ColorThree}{named}{Turquoise}
\tikzstyle{descript} = [text = black,align=center, minimum height=1.8cm, align=center, outer sep=0pt,font = \footnotesize]
\tikzstyle{activity} =[align=center,outer sep=1pt]
\usetheme{Cesbio}
\author{Juan Sebastian Vinasco Salinas}
\date{\today}
\title{Aprendizaje Supervisado}
\subtitle{Una introducción}
\hypersetup{
 pdfauthor={Juan Sebastian Vinasco Salinas},
 pdftitle={Aprendizaje Supervisado},
 pdfkeywords={},
 pdfsubject={},
 pdfcreator={Emacs 30.1 (Org mode 9.7.19)}, 
 pdflang={English}}
\usepackage[backend=bibtex,style=ieee]{biblatex}
\addbibresource{/home/juanse/Documents/Proyectos/IGAC_Diplomado/referencias/referecias_diplomado.bib}
\begin{document}

\maketitle
\begin{frame}{Outline}
\tableofcontents
\end{frame}

\section{Introducción al Aprendizaje Supervizado}
\label{sec:org62dde18}

\begin{frame}[label={sec:orga45fa79}]{Introducción}
\end{frame}
\begin{frame}[label={sec:org69637af}]{Concepto de discretización}
\end{frame}
\begin{frame}[label={sec:orgaea1442}]{Categorización de Modelos de Aprendizaje Supervisado}
\end{frame}


\begin{frame}[label={sec:orgecee8ea}]{Funciones de Activación}
\end{frame}
\section{Ejemplos de modelos clasicos}
\label{sec:orgf5c8984}

\begin{frame}[label={sec:orgc9d83b7}]{Bosque Aleatorio}
\end{frame}
\begin{frame}[label={sec:org8b61b60}]{Redes Neuronales Convolucionales}
\end{frame}
\begin{frame}[label={sec:org5109a59}]{Redes Recurrentes}
\end{frame}
\section{Ejemplos avanzados}
\label{sec:org9084ba0}
\begin{frame}[label={sec:org7d51e26}]{LSTM}
\end{frame}
\begin{frame}[label={sec:orgc65c772}]{Transformers en SSL}
\end{frame}
\section{Ejercicio Practico}
\label{sec:org9f2b16f}

\begin{frame}[label={sec:orgc0e805b}]{Ejercicio Practico}
\begin{figure}{c}
\centering
\includegraphics[width=0.8\textwidth]{./Artwork/car_typing.jpeg}
\caption{\label{fig:gato}Vamos a practicar!!!}
\end{figure}
\end{frame}
\section*{Referencias}
\label{sec:org752391c}
\printbibliography[heading=none]

\newpage
\end{document}
