% Created 2025-09-12 vie 08:34
% Intended LaTeX compiler: pdflatex
\documentclass[11pt]{article}
\usepackage[utf8]{inputenc}
\usepackage[T1]{fontenc}
\usepackage{graphicx}
\usepackage{longtable}
\usepackage{wrapfig}
\usepackage{rotating}
\usepackage[normalem]{ulem}
\usepackage{amsmath}
\usepackage{amssymb}
\usepackage{capt-of}
\usepackage{hyperref}
\author{Juan Sebastián Vinasco Salinas}
\date{\today}
\title{Guía de instalación Orfeo Toolbox y OTBTF}
\hypersetup{
 pdfauthor={Juan Sebastián Vinasco Salinas},
 pdftitle={Guía de instalación Orfeo Toolbox y OTBTF},
 pdfkeywords={},
 pdfsubject={},
 pdfcreator={Emacs 30.2 (Org mode 9.7.19)}, 
 pdflang={English}}
\usepackage{biblatex}

\begin{document}

\maketitle
\tableofcontents

\section{Guía de instalación de OTB y OTBTF}
\label{sec:org9bb0def}

Orfeo Toolbox conocido también por sus siglas OTB, es un software de procesamiento digital de imagenes de satelite, enfocado en procesar imagenes de muy alta resolución.

OTBTF es una extensión de orfeo toolbox que permite la conexion entre OTB y Tensor Flow (en adelante TF), extendiendo las capacidades intrinsecas de OTB a los flujos de trabajo que implican el uso de algoritmos de aprendizaje profundo.
\subsection{Instalación de OTB}
\label{sec:org33fbd98}
\subsubsection{Prerequisitos}
\label{sec:org635b83c}

Ninguno
\subsubsection{Procedimiento de instalación}
\label{sec:orgf62a239}

\begin{enumerate}
\item Dirigase a la pagina de descargas oficial de Orfeo Toolbox \href{https://www.orfeo-toolbox.org/download/}{aquí}
\item Seleccione el sistema operativo y descargue el archivo respectivo
\item Luego, proceda a descomprimir el archivo .zip en el caso de windows
\item Abra una terminal de Powershell en la carpeta donde descompromio el archivo
\item Habilite otb mediante la ejecución del script \emph{otbenv.ps1}
\end{enumerate}
\subsection{Instalalción de OTBTF}
\label{sec:orgd379b17}

\subsubsection{Prerequisitos}
\label{sec:orge4c4429}

Para comenzar necesitamos instalar docker y docker compose, estos programas permiten la descarga, ejecución y administrador de contenedores \footnote{La guía de instalación se basa en la siguiente \href{https://medium.com/@piyushkashyap045/comprehensive-guide-installing-docker-and-docker-compose-on-windows-linux-and-macos-a022cf82ac0b}{guía}}

Para ello sigue los siguientes pasos:

\begin{enumerate}
\item Dirigase a la siguiente dirección web \href{https://www.docker.com/products/docker-desktop/}{docker-desktop}
\item Seleccione el sistema operativo, usualmente es windows AMD64
\item Instale el archivo descargado de extensión.exe
\item Durante la instalación seleccione la opción de habilitar \textbf{wsl2}
\item Una vez finalice la instalación \textbf{reinicie} su computadora
\item Verifique su instalación, lanzando una terminal y ejecutando los siguiente:

\begin{verbatim}
        docker --version
        docker-compose --version
\end{verbatim}
\end{enumerate}
\subsubsection{Procedimiento de instalación}
\label{sec:org401ebce}

Seguiremos el procedimiento estandar de la documentación \footnote{\url{https://otbtf.readthedocs.io/en/latest/docker\_use.html}}

\begin{enumerate}
\item Abra una terminal de powershell
\item ejecute el siguiente comando para una instalación de CPU pura

\begin{verbatim}
        docker pull mdl4eo/otbtf:latest
\end{verbatim}
\item O el siguiente comando si dispone de una GPU

\begin{verbatim}
        docker pull mdl4eo/otbtf:latest-gpu
\end{verbatim}
\end{enumerate}
\end{document}
